\capitulo{2}{Objetivos del proyecto}

En este apartado se va a comentar los objetivos funcionales, técnicos y personales que se tienen en el desarrollo de este proyecto.

\section{Objetivos funcionales}
En este subapartado se va a comentar los distintos objetivos que se querían cumplir con el desarrollo del proyecto:
\begin{itemize}
	\item Desarrollar una aplicación web para poder realizar y evaluar la rehabilitaciones \textit{online} de los pacientes con Parkinson.
	\item Desarrollar una aplicación web accesible y fácil de utilizar, sobre todo para los pacientes.
	\item Crear un herramienta capaz de comparar las rehabilitaciones hechas por los pacientes con los ejercicios bases.
\end{itemize}

\section{Objetivos técnicos}
En este subapartado se va a comentar los distintos objetivos relacionados con las técnicas y herramientas que se quieren aprender y utilizar:
\begin{itemize}
	\item Crear una aplicación web para realizar rehabilitaciones \textit{online}.
	\item Crear una herramienta de comparación de ejercicios a partir de algoritmos de visión por computador y detección de movimientos.
	\item Utilizar el lenguaje de programación \textit{Python} para la comparación de ejercicios.
	\item Realizar el desarrollo del proyecto utilizando un repositorio \textit{Git}, en concreto en \textit{GitHub} para poder controlar las tareas y las versiones del proyecto.
	\item Utilizar la extensión de \textit{Git} llamada \textit{ZenHub} para controlar el estado de las tareas y la temporalidad de estas.
	\item Seguir el modelo \textit{SCRUM} para desarrollar el proyecto de forma incremental.
\end{itemize}

\section{Objetivos personales}
Por último, se van a comentar los objetivos personales que se tienen en este proyecto, donde se encuentran objetivos desde probar conocimientos adquiridos en el máster como mejores algunas cualidades personales.
\begin{itemize}
	\item Poder ayudar a las personas mayores con Parkinson para que puedan realizar las rehabilitaciones necesarias sin necesidad de salir de sus casas. Además, mejorando esta rehabilitación a partir de la comparación de los ejercicios realizados.
	\item Mejorar mis capacidades comunicativas y de exposición en las diversas presentaciones del proyecto a los responsables.
	\item Usar los conocimientos adquiridos duran la carrera y duran el máster.
	\item Mejorar mis conocimientos sobre visión por computador.
	\item Conocer los distintos algoritmos de comparación y clasificación de instancias para los datos obtenidos sobre los ejercicios.
\end{itemize} 

