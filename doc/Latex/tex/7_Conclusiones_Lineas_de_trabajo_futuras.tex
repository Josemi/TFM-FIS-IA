\capitulo{7}{Conclusiones y Líneas de trabajo futuras}

En este apartado se encuentran las conclusiones finales del proyecto, así como el conjunto de tareas y/o mejoras que se podrían realizar.

\section{Conclusiones}
El proyecto se ha desarrollado en una situación anómala. Con la irrupción del \textit{COVID-19} y el correspondiente confinamiento se limitó mucho el desarrollo orientado al paciente que se estaba dando en las primeras fases del desarrollo de la aplicación de rehabilitación \textit{online}. Con suerte, para mediados de febrero ya existía una primera versión de esta aplicación y se pudo realizar, antes de cualquier indicio del virus en el país, la primera instalación de los dispositivos en casa del primer paciente (aunque por problemas con la red \textit{WiFi} se tuvo que quitar).

La imposibilidad de realizar instalaciones de dispositivos impidió la recolección de datos (vídeos) con los que poder implementar un modelo propio de visión artificial. Aunque al realizar el estudio del estado del arte sobre las herramientas existentes se vio que hubiese sido muy complicado la creación de uno de estos modelos principalmente por limitaciones en el \textit{hardware}. Aun así, el estudio del estado del arte realizado ha dado muy buenos resultados, encontrando una herramienta muy buena que permite realizar las tareas de visión por computador en unos tiempos muy buenos.

A partir del estudio de las herramientas y con la obtención del modelo final se pudo estudiar e implementar un sistema de extracción de características de la salida del modelo, con el que posteriormente se desarrolló la comparación de posiciones. Aun con la dificultad de no poder realizar reuniones ni charlas presenciales para comentar el proceso del desarrollo con el equipo médico, sí que se realizaron reuniones de manera \textit{online} y se recibieron vídeos de ejercicios de ejemplos en los que se basaron principalmente para realizar la comparación de posiciones. 

Con todas estas tareas realizadas se cumplió con todos los objetivos marcados al principio del proyecto, incluso superando con creces algunas expectativas sobre el funcionamiento del estimador de posiciones y de los tiempos de ejecución en un problema tan complejo y con una gran cantidad de datos muy pesados.

Actualmente, con la vuelta a la normalidad se está volviendo a realizar instalaciones de dispositivos con los cuales se están recogiendo los primeros datos. Gracias al proyecto, los pacientes con \textit{Parkinson} no necesitarán desplazarse para realizar sus rehabilitaciones y podrán realizarlas de una manera más frecuente con el sistema de retroalimentación de sus ejercicios y evoluciones. Además, con la implementación extensible realizada con la configuración de las comparaciones estas implementaciones se puede utilizar para distintos tipos de ejercicios y de enfermedades. 

Por último, cabe destacar el interés por parte de la OTRI-Transferencia (servicio de la Universidad de Burgos centrado en el desarrollo de ideas y del fomento del emprendimiento en los estudiantes) al recibir el premio prototipo de la edición 2019-2020\footnote{Resolución Premios Prototipo 2019-2020: \href{https://www.ubu.es/te-interesa/convocatoria-prototipos-orientados-al-mercado-curso-2019-2020}{https://www.ubu.es/te-interesa/convocatoria-prototipos-orientados-al-mercado-curso-2019-2020}.}. Gracias a este premio se pudo asistir a distintos talleres de comunicación y presentación de proyectos, marketing digital y de propiedad industrial. Además, se recibió un asesoramiento sobre estos ámbitos.

\section{Líneas de trabajo futuras}
Como se ha comentado, la principal línea de trabajo futura, que ya se está trabajando pero que no entra en este Trabajo Fin de Máster, es la incorporación del proyecto del compañero José Luis Garrido Labrador y este proyecto a la aplicación \textit{web}. Para así permitir a los pacientes realizar ejercicios de forma autónoma.

Aun así, las principales líneas futuras de trabajo orientadas al desarrollo único de este proyecto son:
\begin{itemize}
	\item Poder probar a crear un modelo propio de visión artificial. Por insuficiencia de datos esta meta no se pudo si quiera plantear.
	\item Contar con un mayor presupuesta que permita obtener cámaras de mayor calidad como pueden ser las \textit{Kinect V2} que permiten la obtención de la posición en 3D y mejores ordenadores en las casas de los pacientes que permitan realizar un preprocesado o incluso la extracción de la posición. Convirtiendo así el flujo de vídeos en un flujo de posiciones con un tamaño considerablemente menor, lo que permitiría mejorar los tiempos del flujo aun más.
	\item Extraer mayor información más allá de la comparación de ejercicios, como puede ser el tiempo de reacción de los pacientes.
	\item Implementar en paralelo un predictor de caídas. Los pacientes de \textit{Parkinson} suelen sufrir de caídas debido a las limitaciones en su sistema motor. Por ello un sistema capaz de predecir la caída del paciente para poder detener el ejercicio o incluso alertar a algún familiar o conocido.
\end{itemize}

