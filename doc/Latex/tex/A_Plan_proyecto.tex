\apendice{Plan de Proyecto Software}

\section{Introducción}
La planificación de un proyecto es una fase esencial en desarrollo de éste, ya que permite comprobar y guiar el proyecto según las necesidades actuales y futuras.

Dentro de este plan para el desarrollo del proyecto se pueden diferenciar dos puntos sobre los que se puede enfocar este estudio:
\begin{itemize}
	\item \textbf{Temporal}: enfoque en el que se analiza la evolución del proyecto en el tiempo. Al utilizar una metodología SCRUM, se ha divido el desarrollo en \textit{sprints} de 2 semanas.
	\item \textbf{Viabilidad}: comprobación de la adecuación del proyecto tanto desde el ámbito económico como desde el ámbito legal.
\end{itemize}
\section{Planificación temporal}
Como ya se ha comentado, la planificación temporal se ha divido en distintos \textit{sprints} de 2 semanas cada uno, en lo cuales se realizaron distintas tareas y reuniones. Cabe destacar que antes de realizar el primer \textit{sprint} se desarrolló la aplicación FIS sobre la cuál será implementado en el futuro el proyecto desarrollado.

\subsection{Desarrollo de la aplicación FIS}
¿Esto lo documento mas?
\subsection{\textit{Sprint}1 - 17/02/2020 -> 26/02/2020}
Este es el primer \textit{sprint} de desarrollo de la aplicación, en el se realizaron las primeras tareas de creación del repositorio en \textit{GitHub} y la creación de la estructura del mismo. Además, se creo el documento \textit{LaTeX} a partir de la plantilla de la asignatura. Las tareas más relevantes en este primer \textit{sprint} fueron las relacionadas con la configuración de \textit{Gamma} y los primeros pasos en la investigación de algoritmos de visión por computador y detección de movimiento, como son \textit{Detectron2} y \textit{PoseNet}.
\section{Estudio de viabilidad}

\subsection{Viabilidad económica}

\subsection{Viabilidad legal}


